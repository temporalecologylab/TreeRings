% !TeX root = paper.tex
\documentclass[a4paper,12pt]{article}

% Packages
\usepackage[utf8]{inputenc}     % For UTF-8 encoding
\usepackage{amsmath, amssymb}   % For math symbols
\usepackage{graphicx}           % For including graphics
\usepackage{geometry}           % For setting page margins
\usepackage{natbib}             % For citations
\usepackage{hyperref}           % For clickable links in references

% Page layout
\geometry{margin=1in}           % Set margins to 1 inch

% Title section
\title{Tina Paper In Development}
\author{Your Name}
\date{\today}

\begin{document}

\maketitle

\begin{abstract}
    This is the abstract of your paper. Briefly describe the purpose of the research, the main results, and the conclusions.
\end{abstract}

\section{Introduction}
This is the introduction section where you introduce the topic and cite some papers \citep{mohammadi_comparative_2024}.

\section{Methods}
Describe your methods here. You can use equations like this:

\subsection{Hardware Design}

\subsubsection{Mechanical Structure}
The cartesian robot was purchased from an open source parts supplier, OpenBuilds. They sell various sets of gantry robots, motor controllers, and power supplies
which can be conveniently purchased from one supplier. All metal components are cut and sized, and have very detailed assembly videos, an active community forum,
and customer support to fall back on. The OpenBuilds ACRO system circumvents the need for most power tools and machining that would otherwise be necessary to build a robust robot.
All other parts were designed using SolidWorks and printed on an FDM 3D printer using ABS material. Mechanical assembly can be completed entirely with common hand tools and metric fasteners. 

\subsubsection{Computer}
The computer which runs the graphical user interface (GUI), gantry control, camera streaming, and image stitching is the NVIDIA Jetson Orin Nano Developer Kit (Orin Nano). Using a computer
which is compatible with CSI cameras such as the Raspberry Pi HQ Camera was essential to maintain cost effectiveness. The Orin Nano, like others in the Jetson line of products, has hardware
accelerators for common GPU tasks such as image streaming, which is vital to retain CPU power for GUI tasks and robot control. 

\subsubsection{Camera}
Unlike the Gigapixel or CaptuRING, the camera is a machine vision camera rather than a handheld professional camera. Choosing the Raspberry Pi HQ Camera provided many lens options, significant
forum support, and includes an Obsolescence Statement that the camera will be in production until January 2030. Combined with a 180mm C-mount microscope lens, the images can reach above 20,000 dots per inch (DPI).

\subsection{Software Design}
All software was written in Python VERSION and a variety of packages for the GUI, image stitching, and calculations (Should I include all the packages in writing?). Care was taken to segment the
software design into abstracted object oriented code. 

\subsubsection{Graphical User Interface}
The GUI to control the machine launches from a Python script. Included are a viewer to see the live stream from the camera, buttons to jog the machine throughout the XYZ coordinate system,
setting the size of samples, and running the process to capture and stitch images. The system was designed to allow for multiple samples to be prepared in a queue for bulk digitization. 

\subsubsection{Sample Digitization}
The digitization of a sample can be reduced to a few processes - capturing a grid of overlapping images, image focusing, and image stitching. 

\section{Results}
Present your results here.

\subsection{Scans of Cookies and Cores}

\subsection{Functional Limits}

\section{Discussion}
Discuss the implications of your results here.

\subsection{Strengths and Opportunities}

\subsection{Opportunities for Improvement}

\section{Conclusion}
Summarize your key findings here.

\bibliographystyle{plainnat}
\bibliography{references}

\end{document}
