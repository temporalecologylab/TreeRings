\begin{table}[h]
    \centering
    \caption{Mathematical notation used in this paper.}
    \label{tab:math_notation}
    \begin{tabular}{ |p{3cm}||p{12cm}|  }
      \hline
      \multicolumn{2}{|c|}{Math Notation} \\
      \hline
      Terms & Definition\\
      \hline
      $X_0, Y_0, Z_0$   &The initial cartesian coordinates of the camera when the sample is saved by the user. \\
      $(X_k, Y_k, Z_k)$ & A unique set of $X$, $Y$, and $Z$ coordinates for the center point of an individual image $k$. \\
      $(X_{k+1}, Y_{k+1})$ & The $X$ and $Y$ coordinates of an image that neighbors image $k$ at $(X_k, Y_k)$. \\
      $\boldsymbol{\delta}$ & Equally spaced heights, relative to $Z_0$, to capture images at and use to find a sharp image. \\
      $\alpha_k$ & The response variable of the PID control algorithm which adjusts the initial height of an image to gradually align $Z_{focused}$ to be at the center index of $\boldsymbol{Z_{samples}}$.\\
      $NV()$ & Normalized variance 'function.' \\
      $Z_{focused}$ & The $Z$ value of the most in-focus image in $\boldsymbol{Z_{\text{samples}}}$. \\
      $\boldsymbol{Z_{\text{samples}}}$ & A vector containing the $Z$ coordinate for multiple images at the same $X_k, Y_k$. \\
      $i_{\max,k}$ & The index of the image with the largest normalized variance at the same $(X_k, Y_k)$. \[
      i_{\max} = \arg\max_{i} NV(image(\boldsymbol{{Z_\text{samples}})})
      \] \\
      $w$ & The width of the window to traverse when trying to center a core. \\
      $image()$ & The image at a corresponding $Z$ value from $\boldsymbol{Z_{\text{samples}}}$. \\
      $d_1$ & The relative distance from the camera to one edge of a leveled sample. \\
      $d_2$ & The relative distance from the camera to the opposite edge of a leveled sample. \\
      \hline
    \end{tabular}
    \end{table}