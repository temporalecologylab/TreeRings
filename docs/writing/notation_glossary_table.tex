\begin{table}[htbp]
    \centering
    \label{tab:math_notation}
    \begin{tabular}{ |p{3.5cm}||p{11.5cm}|  }
      \hline
      % \multicolumn{2}{|c|}{Math Notation} \\ %emw14Apr2025 -- remove this line -- you just need the caption and then the table should start immediately
      \hline
      Terms & Definition\\ 
      \hline
      $X_0, Y_0, Z_0$   &The initial cartesian coordinates of the camera when the sample is saved by the user. \\
      % $Z_k$ & The center height of $\boldsymbol{Z_k}$ which was the best approximation of $Z_{focused,k}$ from the previous image.\\ 
      % $(X_k, Y_k, Z_{focused,k})$ & A unique set of $X$, $Y$, and $Z$ coordinates for a unique image of the sample, $k$. \\
      % $(X_{k+1}, Y_{k+1})$ & The $X$ and $Y$ coordinates of an image that neighbors image $k$ at $(X_k, Y_k)$. \\
      % $\boldsymbol{\delta}$ & Equally spaced heights, relative to $Z_k$, to capture images at and use to find a sharp image. \\
      % $\alpha_k$ & The response variable of the PID control algorithm which adjusts the initial height of an image to gradually align $Z_{focused,k}$ to be at the center index of $\boldsymbol{Z_k}$.\\
      % $Z_{focused,k}$ & The $Z$ value of the most in-focus image in $\boldsymbol{Z_k}$. \\
      % $Z_{focused,0}$ & The $Z$ value of the center coordinate, and first image of the sample. \\
      % $\boldsymbol{Z_k}$ & A vector containing the $Z$ coordinates for multiple images at the same $(X_k, Y_k)$. \\
      % $w$ & The width of the window to traverse when trying to center a core. \\
      $d_1$ & The relative distance from the camera to one edge of a leveled sample (Figure \ref{fig:sample_levelling}).\\
      $d_2$ & The relative distance from the camera to the opposite edge of a leveled sample (Figure \ref{fig:sample_levelling}). \\
      \hline
    \end{tabular}
    \caption{Mathematical notation used in this paper.} %emw14Apr2025: Explain what the bolding means
    \end{table}
    