\documentclass{article}
\usepackage[top=1.00in, bottom=1.0in, left=1in, right=1in]{geometry}
\usepackage{outlines}
\usepackage{enumitem}
\setenumerate[1]{label=\Roman*.}
\setenumerate[2]{label=\Alph*.}
\setenumerate[3]{label=\roman*.}
\setenumerate[4]{label=\alph*.}
\begin{document}
\section {Paper Outline}
\begin{outline}[itemize]

\1 Abstract
\begin{itemize}
	\item  Digital twins of samples is important for building data repositories and promoting replicable science
	\item  Digitizing wood samples has limits to sample dimensions or lacks affordable implementations
	\item 	Current alternatives are very expensive (\$70,000USD), lack a high level of detail (printer scanners), or cannot scan cookies (CaptuRING)
\end{itemize}
\item Introduction
	\begin{itemize}
	\item Tree ring science overview
		\begin{itemize}
		\item **Lizzie add here**
		\item tree rings can help reconstruct past climates 
		\item 
		\end{itemize}
	\item Development of measurement and recording systems
	\item Digitizing wood samples for better data pooling / archiving
	\item Software can be used to help record annual growth 
	\item Main reasons 
		\begin{itemize}
		\item affordable
		\item open source
		\item  alternative to scanners, without size limitations to digitize tree cookies and cores 
		\end{itemize}
	\item Provide a means to study tree growth from more detailed data than tree cores alone
	\item Development of measurement and recording systems
	\end{itemize}
\item Materials and Methods
	\begin{itemize}
	\item System Design 
		\begin{itemize}
		\item Hardware
			\begin{itemize}
			\item Cartesian gantry robot
			\item Camera
			\item computer
			\item camera and computer mount
			\end{itemize}
		\item Software
			\begin{itemize}
			\item image focusing
			\item grid traverse
			\item feature based image stitching
			\subitem memory mapped 
			\item user interface
			\end{itemize}
		\end{itemize}
	\end{itemize}
\item Results
	\begin{itemize}
	\item ultra high resolution scans (DPI 15,000 +)
	\item scans of cookies / cores
	\item  Digitization time as a function of surface area
	\item File-size results in functional limits 
		\begin{itemize}
		\item max filesize for TIFF files is 2.5 GB
		\item other lossless filetypes for larger images are not compatible with standard image viewers
		\item viewing a pseudo-core of a larger image and cropping 
		\item downscaling
		\end{itemize}
	\end{itemize}
\item Discussion (for MEE Only)
	\begin{itemize}
	\item strengths and opportunities
		\begin{itemize}
		\item  Ability to capture multiple cores / cookies simultaneously
		\item cost effective
		\item potential for automatic tree ring identification with iamge processing / ML 
		\item potential for vessel counts for an entire ring 
		\subitem although final quality is only as good as teh sample preparation
		\end{itemize}
	\item opportunities for improvements
		\item getting a better lens
			\begin{itemize}
			\item increases sharpness
			\item improves stitching accuracy
			\item obtain similar quality between the corner and center of the image
			\subitem poor lens quality seemed to be the culprit of producing visible seams when stitching with larger field of view images 
			\end{itemize}

		\item focus stacking
		\subitem could allow for non planar objects to be digitized and stitched
		\item Difficulty stitching with lower detail images, can't increase digitizing speed, must downscale high resolution images 
		\item Autofocusing with hardware would significantly decrease the time to capture a cookie (using a professional camera)
	\end{itemize}
\item Conclusions (optional for MEE)

        
\end{outline}
\end{document}