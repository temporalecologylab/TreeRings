\documentclass{article}
\usepackage[top=1.00in, bottom=1.0in, left=1in, right=1in]{geometry}
\usepackage{outlines}
\usepackage{enumitem}
\usepackage{graphicx}
\setenumerate[1]{label=\Roman*.}
\setenumerate[2]{label=\Alph*.}
\setenumerate[3]{label=\roman*.}
\setenumerate[4]{label=\alph*.}
\begin{document}
\section {Paper Outline}
\begin{outline}[enumerate]

\1 Abstract
\begin{enumerate}
	\item Flatbed scanner alternative with improved resolution to digitizing tree cookies and cores
	\item Digital twins of samples is important for building data repositories and promoting replicable science
	\item Digitizing wood samples has limits to sample dimensions or lacks affordable implementations
	\item Current alternatives are very expensive (\$70,000USD), lack a high level of detail (printer scanners), or cannot scan cookies (CaptuRING)
	\item Tina is open source and able to be assembled with 3D printed parts, OpenBuilds parts kits, and hand tools % Can we add how much ours costs?
\end{enumerate}
\item Introduction
	\begin{enumerate}
	\item Tree ring science overview
		\begin{enumerate}
		\item tree rings provide critical data/insights across multiple fields % Guibal and Guiot
		\item tree rings can help reconstruct past climates (dendroclimatology) % Sheppard
		\item understanding tree growth mechanisms (plant bio/ecology) % Fritts and Swetnam 
		\end{enumerate}
	% Consider adding this to below points (Development of scanning alternatives) as one paragraph on its own
	\item Getting these data means measuring tree rings -- quick history
		\begin{enumerate}
		\item originally done on scopes (stage micrometer) % Robinson and Evans, https://conservancy.umn.edu/server/api/core/bitstreams/30f91f03-c27b-44de-8659-5156220ce79b/content
		\item eventually there was a shift to analyzing digital images of samples instead of stage micrometers % Guay et al. 1992 
		\item digital copies inherently allow for contributing to data pools  % Rademacher​ et al. WIAD; also International Tree-Ring Data Bank (IRTB) https://www.ncei.noaa.gov/products/paleoclimatology/tree-ring
		\item Development of image acquisition in dendrochronology 
			\begin{enumerate}
			\item two models: flatbed scanner or overlapping images and stitch
			\item beginning with flatbed scanners % WinDendro
				\subitem modern flatbed scanners can achieve high resolution (4200 dpi) but size of samples is limited % see CaptuRING paper for flatbed specs
				\subitem can scan both cookies and cores	
			\item overlapping images and stitching 
				\subitem ATRICS, Gigapixel, CaptuRING 
				\subitem so far can only capture cores
				\subitem budget friendly implementations require user intervention between image acquisition and stitching
				\subitem takes advantage of advanced camera technology
			\end{enumerate}
		% \item Could add here one sentence on the three methods you mention below depending on what you want to say ... 
		\end{enumerate}
	
	\item Main reasons to build this 
		\begin{enumerate} 
		\item further develop the affordable / open source alternatives for scientific equipment
		\item extend the capabilities of CaptuRING by allowing to stitch cookies and multiple samples in a queue
		\item reduce the need for proprietary software such as PTGUI for image stitching
		\item increase the maximum size of a sample 
		\item Digitizing wood samples for better data pooling / archiving
		\end{enumerate} 
	\item Here we tell you about our new device
		\begin{enumerate}
		\item We review hardware and software 
		\item total cost
		\item Then show how the method can be used to record annual growth through current common programs 
		\item allows for multiple cookies / cores to be added to a queue for sampling
		\item stitching on device 
		\end{enumerate}
	\end{enumerate}
\item Materials and Methods
		\begin{enumerate}
		\item Guiding design considerations
			\begin{enumerate}
			\item Digitization of a tree cookie and core
			\item Minimizing barriers to assemble (Cost effective, open source, open hardware, minimal use of specialized equipment/tools)
			\item Reducing user interaction between sample prep and the final image
			\item Achieving near microscopic detail with a field of view as large as the sample
			\end{enumerate}
		\item Considerations for Digitizing Cookies
			\begin{enumerate}
			\item Inconsistencies in cookie preparation, (different height, non parallel faces)%inconsistent_cookies.jpg
				\subitem figure for inconsistencies
			\item Cookie's surface area requires movement in the X and Y direction
			\item The inconsistencies require movement in the Z direction to maintain in focus images
			\end{enumerate}
		\item Hardware Design
			\begin{enumerate}
			\item Biggest cost in most alternatives is a professional camera, choosing a Raspberry Pi HQ camera avoids this. The camera also has a definitive obscelescence statement %camera 
			\item But to use a CSI camera, we need special hardware. NVIDIA Jetson Orin Nano fits the bill and has enough computational power to handle the image processing tasks while powering a monitor and GUI %computer 
			\item Ideally all non-structural components could be purchased off the shelf and assembled. Fortunately complex geometries, dimensional accuracy, quick iteration speed, and cost effectiveness can be achieved with 3D printing  % 3d printing
				\subitem Parts can easily be ordered from 3D printing shops but also has a low technical barrier for personal use
			\item to combat sample inconsistencies, a levelling table and bullseye level were designed to make the prepared surface of the cookie closer to parallel to the XY plane
			\item for the structure, the 1m x 1m OpenBuilds ACRO system was used. This kit is made of strong aluminum extrusion, has 4.5 micron resolution of accuracy, and is assembled quickly by following OpenBuild's detailed video instructions. % cartesian gantry machine
			\item For the Z axis a lead screw linear actuator was attached to the mounting plate of the ACRO
			\end{enumerate}
		\item Software Design
			\begin{enumerate}
			\item Everything was designed in Python 
			\item GUI to load samples and their identifiers to the queue to be digitized
			\item image focusing
				\begin{enumerate}
				\item A routine was made to take multiple images at varied Z-heights as a proxy for autofocusing. The image with the highest normalized variance score was chosen as the in focus image and is kept. %reference
				\item Rather than moving the Z-axis, taking a photo, then moving again, the images are taken when the Z-axis is moving at constant velocity and are triggered with a time delay. Stopping and starting the Z-axis results in significant vibration blur in the image, requiring 2 to 3 seconds of inactivity after coming to a stop before taking an image
				\item It's possible for the height variation to exceed the range of the multiple images taken. A PID automatic control algorithm was used to make adjustments to the initial Z-height of the image stack. This is done by adjusting to keep the focused image in the center of the image stack % reference to PID
				\item figure for focusing 
				\item figure for sample setup ideal/realistic
				\end{enumerate}
			\item obtaining a large field of view with a grid of small field of view images 
				\begin{enumerate}
				\item When the sample is loaded into the machine, the machine traverses the user defined bounding box of the sample
				\item It automatically takes a grid of images where each image has overlap with its adjacent neighbors
				\item The overlap in the images allows for the images to be aligned and combined to their neighbors. Stitching an image to its neighbors is done by comparing calculated key points on each image and stitching them on their matches.
				\item A python package Stitch2D was made to stitch together a grid from a set of structured images % should I talk about the changes I made to the package here or in the discussion?
					\subitem We made changes to the package's stitching algorithm for memory efficiency as samples with thousands of images needed more RAM than what was available
				\item figure for grid traverse
				\end{enumerate}
			\end{enumerate}
		\end{enumerate}
\item Results
	\begin{enumerate}
	\item scans of cookies / cores
		\begin{enumerate}
		\item ultra high resolution scans (DPI 13,500 +)
		\item downscaled versions as high resolution is not necessary for all applications
		\item Digitization time and file size as a function of surface area. Large cookies can scan only a portion of the cookie including the center and half of the rings.
		\item large surface areas can produce files that are inconveniently large
		\end{enumerate}
	\item File-size results in functional limits 
		\begin{enumerate}
		\item max filesize for TIFF files is 2.5 GB
		\item other lossless filetypes for larger images are not compatible with standard image viewers
		\item workaround with NumPy memory maps and cropped viewing 
		\end{enumerate}
	\item Images/tables	
		\begin{enumerate}
		\item Scale test image, proves the generated DPI calculation negates need to image with a ruler in frame
		\item Sample size vs image capturing time graph
		\item Hopefully sample size, dpi, automated tree ring time
		\item Table with results of Equation for uncompressed filesize, emphasize lossy vs lossless compression, and true filesize %image filesize = DPI x 25.4 x sample-h-mm x sample-w-mm x 8bits x 3 colors
		\end{enumerate}
	\end{enumerate}
\item Discussion (for MEE Only) 
	\begin{enumerate}
	\item Quick overview (TODO)
		\begin{enumerate}
		\item X,Y,Z imaging machine which acts as a scanner 
		\item obtains a grid of in focus images and slightly overlapping images which get stitched together to form one large mosaic 
		\item designed for other labs with minimal engineering experience or equipment to build this for themselves
		\end{enumerate}
	\item strengths and opportunities
		\begin{enumerate} % Usually I like a quick overview paragraph and I think you should still have one but as currently written I think the 'potential for automatic tree ring identification with iamge processing / ML ' and 'potential for vessel counts for an entire ring' connect too much to later sections, so you should move those sections up 
		\item Ability to capture multiple cores / cookies simultaneously
		\item cost effective
		\item vessel counting requires detail at the sub-vessel level % https://roxas.wsl.ch/en/detail is only as good as sample prep
		\item tested on CooRecorder + CDendro
			\begin{enumerate}
			\item Maximum filesize 
			\item Coordinates were registered on to the cookie scans 
			\item Note to self: using CDendro for multiple samples from the same tree... (cookies) 
			\end{enumerate}
		\item tested on R-CNN for deep learning tree ring identification
			\begin{enumerate}
			\item potential for integration with a more powerful computer to automatically identify rings while imaging
			\item potential for final stitched images to have rings automatically counted
			\item Cookie with a dpi of SMALL and dpi of 13500 were tested. With total pixel count of SMALL and BIG
				\subitem NEEDS specific: DPI, height and width of pixels, record runtime
				\subitem NEEDS specifications of server computational power (GPU count/other metric for ANN power)
			\end{enumerate}
		\end{enumerate}
	\item opportunities for improvements %combine into strengths and opportunities? 
		\begin{enumerate}
		\item Software support for large files than 2.5GB >
			\begin{enumerate}
			\item Not all software can support this, especially if the computer is RAM constrained. Ring analysis tool which uses BigTIFF, HD5, or memory maps and loads only partial portions of the full image would be useful
			\end{enumerate}
		\item Integration with automatic tree ring identification model
			\begin{enumerate}
			\item It would be interesting to analyze the images before stitching in a different thread to potentially stitch together a binary bitmap which highlight tree rings
			\item This could potentially drop the need for a hypercomputer if inference could be made concurrently with the images being captured
			\end{enumerate}
		\item Identifying vessel counts in each growth year 
		\end{enumerate}
\item Conclusions (optional for MEE) % Usually I am anti-conclusions because you just end up repeating yourself, but I think this could work here, especially to end on a positive after the 'opportunities' section. I think you could add  the 'potential for automatic tree ring identification with iamge processing / ML ' and 'potential for vessel counts for an entire ring'  here perhaps.
	\begin{enumerate}
	\item empty
	\end{enumerate}

\end{enumerate}
\end{outline}
\end{document}