\documentclass{article}
\usepackage[top=1.00in, bottom=1.0in, left=1in, right=1in]{geometry}
\usepackage{outlines}
\usepackage{enumitem}
\setenumerate[1]{label=\Roman*.}
\setenumerate[2]{label=\Alph*.}
\setenumerate[3]{label=\roman*.}
\setenumerate[4]{label=\alph*.}
\begin{document}
\section {Paper Outline}
\begin{outline}[itemize]

\1 Abstract
\begin{itemize}
	\item Flatbed scanner alternative with improved resolution to digitizing tree cookies and cores
	\item Digital twins of samples is important for building data repositories and promoting replicable science
	\item Digitizing wood samples has limits to sample dimensions or lacks affordable implementations
	\item 	Current alternatives are very expensive (\$70,000USD), lack a high level of detail (printer scanners), or cannot scan cookies (CaptuRING)
	\item Tina is open source and able to be assembled with 3D printed parts, OpenBuilds parts kits, and hand tools
\end{itemize}
\item Introduction
	\begin{itemize}
	\item Tree ring science overview
		\begin{itemize}
		\item **Lizzie add here**
		\item tree rings can help reconstruct past climates 
		\item understanding tree growth mechanisms
		\end{itemize}
	\item Development of scanning alternatives
		\begin{itemize}
		\item beginning with ATRICS
		\item expensive Gigapixel
		\item DIY CaptuRING
		\end{itemize}
	\item Main reasons to build this 
		\begin{itemize}
		\item further develop the affordable / open source alternatives for scientific equipment
		\item reduce the need for proprietary software such as PTGUI for image stitching
		\item allow a method to digitize cookies to have higher quality data than cores alone
		\item avoid size limitations to digitize tree cookies and cores 
		\item Digitizing wood samples for better data pooling / archiving
		\end{itemize}
	\item Software can be used to help record annual growth 
		\begin{itemize}
		\item Coorecorder
		\item Various ML based methods that Sandy has researched
		\end{itemize}
	\end{itemize}
\item Materials and Methods
		\begin{itemize}
		\item Hardware Design
			\begin{itemize}
			\item Cartesian gantry robot kit from OpenBuilds (ACRO system)
			\item Camera 
				\begin{itemize}
				\item chose Raspberry Pi HQ Camera for C-mount lens options, ability to stream, cost effectiveness, and clear obscelescence statement
				\end{itemize}
			\item computer
				\begin{itemize}
				\item NVIDIA Jetson Orin Nano for processing power in a light package, can connect to Raspberry Pi camera. Powerful enough to power a GUI and monitor display as well
				\item NVMM Hardware accelerated video streaming reduces CPU load 
				\end{itemize}
			\item 3D printed components
				\begin{itemize}
				\item 3D printed parts from a Bambu X1C were designed to mount the lens and computer to the ACRO system
				\item other parts such as a drag chain for cable management were also 3D printed
				\item levelling tables were designed to allow for the cookie samples to be arranged orthogonally to the lens. Imperative to capturing in focus images across the entire cookie
				\end{itemize}
			\end{itemize}
		\item Software Design
			\begin{itemize}
			\item image focusing
				\begin{itemize}
				\item sample levelling is not perfect. At different points on the cookie, the Z-height of the gantry must vary to capture the in focus image. A routine was made to take multiple images at varied Z-heights. The image with the highest normalized variance score was chosen as the in focus image and is kept.
				\item automatic PID control was implemented to adjust the initial Z-coordinate for each set of images to increase the probability of having an in focus image at each image set
				\end{itemize}
			\item grid traverse
				\begin{itemize}
				\item The software traverses the entire surface area of the sample. Each row and column is overlapping with its adjacent row and column. Allowing for feature based image sttiching.
				\end{itemize}
			\item feature based image stitching
				\begin{itemize}
				\item An implementation, Stitch2D, of feature based OpenCV image stitching was used. Edits to the code were made to allow for more memory efficient methods of stitching with the use of NumPy memory maps
				\end{itemize}
			\item user interface
				\begin{itemize}
				\item A GUI was made to navigate the machine to samples, set the size of the sample and its center. After samples are set up the operation is passive.
				\end{itemize}
			\end{itemize}
		\end{itemize}

\item Results
	\begin{itemize}
	\item scans of cookies / cores
		\begin{itemize}
		\item ultra high resolution scans (DPI 15,000 +)
		\item downscaled versions as high resolution is not necessary for all applications
		\item Digitization time as a function of surface area. Large cookies can scan only a portion of the cookie including the center and half of the rings.
		\item large surface areas can produce files that are inconveniently large
		\end{itemize}
	\item File-size results in functional limits 
		\begin{itemize}
		\item max filesize for TIFF files is 2.5 GB
		\item other lossless filetypes for larger images are not compatible with standard image viewers
		\item workaround with NumPy memory maps and cropped viewing 
		\end{itemize}
	\end{itemize}
\item Discussion (for MEE Only)
	\begin{itemize}
	\item strengths and opportunities
		\begin{itemize}
		\item  Ability to capture multiple cores / cookies simultaneously
		\item cost effective
		\item potential for automatic tree ring identification with iamge processing / ML 
		\item potential for vessel counts for an entire ring 
		\subitem although final quality is only as good as the sample preparation
		\end{itemize}
	\item opportunities for improvements
		\item Lenses
			\begin{itemize}
			\item troubles with fixed aperature / non autofocusing / poorly manufactured lens
				\begin{itemize}
				\item lack of control to increase sharpness
				\item autofocusing with Z-axis movement is time consuming to do without image blur
				\item poor lens manufacturing can lead to drastically different quality / lighting between the edges and center of an image
				\end{itemize}
			\item good lens can increase sharpness and therefore stitching accuracy and ML inference
			\item obtain similar quality between the corner and center of the image
			\subitem poor lens quality seemed to be the culprit of producing visible seams when stitching with larger field of view images 
			\end{itemize}

		\item focus stacking
		\subitem when a sample is microtomed, the vessels are hollow. The autofocusing algorithm can be confused from this 
		\item Difficulty stitching with lower detail images, can't increase digitizing speed, must downscale high resolution images 
		\item Autofocusing with hardware would significantly decrease the time to capture a cookie (using a professional camera)
	\end{itemize}
\item Conclusions (optional for MEE)
	\begin{itemize}
	\item
	\end{itemize}
        
\end{outline}
\end{document}